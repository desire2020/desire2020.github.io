\documentclass{article}
    \usepackage[T1]{fontenc}
    \usepackage[utf8]{inputenc}
    \usepackage[margin=1in]{geometry}
    
    
    \newcommand{\HRule}{\rule{\linewidth}{0.5mm}}
    \newcommand{\Hrule}{\rule{\linewidth}{0.3mm}}
    
    \makeatletter% since there's an at-sign (@) in the command name
    \renewcommand{\@maketitle}{%
      \parindent=0pt% don't indent paragraphs in the title block
      \centering
      {\Large \bfseries\textsc{\@title}}
      \HRule\par%
      \textit{\@author \hfill \@date}
      \par
    }
    \makeatother% resets the meaning of the at-sign (@)
    
    \title{Personal History Statement}
    \author{Sidi Lu}
    \date{Ph.D. Applicant}
    
    \begin{document}
      \maketitle% prints the title block
      \thispagestyle{empty}
      \vspace{35pt}
    
      When I was only a little boy, I always wanted a friend to chat with. Out of innocent greed, I expect he/she to understand every single piece of idea in my head, with endless passion and patience. I kept looking for such a friend, ended up with understanding a fact that, it's hard for humans to understand each other, and that understanding me is not an exception. From then on, I started to imagine that what if there is a machine that does this better than me and anyone else, which can be used to help humans to better communicate with each other. By better understanding what others need and care, everyone would get a chance to live in a better world, where no one shall ever suffer from solitude. During those old days, my adventure with my computer got me really fascinated with this tiny machine. 

      After I'd received my basic education of Chinese and English in my primary school, I was fascinated by the beauty of languages and literature. I started my attempts at writing poems in Chinese and even English (though poor English then) in 2009. Since then, I've wrote over 200 modern Chinese poems, and participated in the publication of a poetry collection as a member of the Bai Yan poetry club, Shanghai Jiao Tong University. On the other hand, after I had a rudimentary grasp of programming skills in the middle school, a new perspective to understand language and human intelligence has gradually formed in my mind. I realize the latent yet strong connections between formal languages (like programming languages) and natural languages (like Chinese and English). I gradually got really fascinated in finding out such connections and expressing it explicitly using programs and natural languages. When I began my exploration in creating poem-writing programs and chatbots then, I found that I was not completely able to handle all the details. It seems that there is much for me to learn and think. Since then, computer science has always been my dream major. 
      
      During my school life in middle school and high school, I spent most of my spare time exploring the world of languages (both formal and natural languages). I participated in the Chinese National Olympiad in Informatics in Provinces (NOIp) and won the first class prize in NOIp 2013 and 2014. It is a true test of my programming skills. Although the burden of schoolwork is heavy, as that most Chinese students would have to face the pressure of the College Entrance Examination, the success in NOIp encouraged me to know more about computer science. After entering Shanghai Jiao Tong University (one of the top 5 colleges in China), I got accepted as a member of the ACM Honors Class because of such a positive recognition of my ability and my enthusiasm for computer science. 

      Being a member of the ACM Honors Class of SJTU is really a challenge to me. The majority of the class have the experience of winning at least the third class prize in the Chinese National Olympiad in Informatics (NOI). Compared to them, my skills are not supporting my pride as a good programmer. In the first semester I really had a hard time, since what I'd thought to be my advantage was no longer outstanding. I was not struck down, as I gradually learned to adapt to the new environment. When I finished my basic courses in college and started to explore the subject and do researches, such ability to adjust myself to new environment helped me a lot.
      
      I took Professor Weinan Zhang's Machine Learning course in the second term of my sophomore year. During the course, all attendants are required to join two in-class competitions. The goal of the first competition is to solve a binary text classification problem. I studied really hard on this task, with a lot of empirical attempts using various statistical machine learning approaches. At the end of the competition, I won the 3rd place out of 34 participants. Another competition is to design a recommendation system. I won the 2nd place out of 31 participants. Because of my outstanding achievements, I was awarded 100/100 for this course. This honor encouraged me to explore more in the field of machine learning.

      After I've finished my sophomore courses, I joined the APEX data and knowledge management lab of SJTU as an undergraduate member, supervised by Professor Weinan Zhang and Professor Yong Yu. During that summer vacation, I had my first research internship in APEX. The atmosphere of communication is quite good in APEX. After each day's lecture, Weinan would chair a brief discussion with lab members, during which students are encouraged to share their comments and thoughts. Such brainstorm encouraged me to think deeper about machine learning, establishing a reliable knowledge base for me to support further research. 

      My experience in APEX is quite joyful, as I am a self-motivating student, and willing to participate in other group's discussion and propose my own ideas. I do not need much attendance and I can do my job well. As a result, I've submitted/published 4 papers in APEX as the first or second author during my life in APEX. My total citation number (statistics from Google Scholar) is now over 45+ and I have a h-index number of 3.

      In the long run, when I finish my graduate research life, I plan to work as a faculty, to share my knowledge and exploration with people. I also hope one day my team and I would be able to automatize the judical system to improve the efficiency and sense of justice in judical processes.
      


    \end{document}
